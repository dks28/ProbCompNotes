\chapter{ The Optional Stopping Theorem}

In this chapter, we investigate the interaction of martingales with stopping times. As an example, 
we shall refer to the Gambler's Ruin process discussed previously. To remind the reader, the 
Gambler's Ruin is a stochastic process $(X_i)_{i\in\mathbb{N}_0}$ defined by $X_0 = k$, for some 
positive integer $k$, and $X_{i+1}$ is distributed uniformly over $\{-1, 1\}$ for any $i \in 
\mathbb{N}_0$. The usual interpretation (leading to the process's common name) is that $X_0$ is a 
starting capital for a gambler, and each subsequent $X_i$ indicates the profit made in the $i$th 
round of play. \\
Overall then, it makes sense to consider the capital the gambler possesses at each time step. 
Denote this by $(S_i)_{i\in \mathbb{N}_0}$, and note that we take $S_i = X_0 + \sum_{j=1}^{j=i}X_i$
for any $i$. Then, it should be reasonably obvious that $(S_i)_{i\in\mathbb{N}_0}$ is a martingale
with respect to $(X_i)_{i\in\mathbb{N}_0}$, since each $X_i$is bounded below and above by $k-i$ and
$k+i$, respectively (and the other axioms are obviously satisfied). 

We shall consider a few stopping strategies (which yield stopping times) for the gambler; for 
example the one we had considered previously, in which the gambler stops playing when either he 
goes bankrupt or when he has $N > k$ pounds. We have already explored his process to some extent, 
for example considering the expected capital the gambler would have by the time he stops playing. 
The main aim of this chapter is to establish whether the gambler could improve his prospects by 
changing his stopping strategy.

\section{Review of Gambler's Ruin Stopping Times}
	Recall that a stopping time for a a stochastic process $(X_i)_{i\in \mathbb{N}_0}$ is a 
	random variable $\tau$ such that any event $\tau = i$ can be defined depending only on $X_0
	\hdots X_i$. \\
	Now, suppose first that the gambler simply intends to play for a total of $n$ rounds, then 
	$\tau \coloneqq \lambda \omega . n$ is obviously a stopping time for 
	$(X_i)_{i\in\mathbb{N}_0}$. In fact, the expected capital of the gambler at time $\tau$ can
	easily be determined as $\mathbb{E}(X_0) = k$; this follows from a simple inductive 
	argument. 

	In fact, this co\"incides with the result we had previously obtained when considering the 
	Gambler's Ruin as a Markov chain, and follwoing a different stopping strategy: By the 
	result of lemma \ref{lemma:grprsucc}, we had that for our original stopping strategy (say 
	$\tau'$), $\mathbb{E}(S_{\tau'}) = 0 \cdot \frac{n-k}{n} + n \cdot \frac{k}{n} = k$.  

	\subsection{Stopped Martingales}
	We would like to be able to combine the ideas of matingales with those of stopping times.
	Luckily, we may simply use a construct which `stops' a martingale at a given stopping time,
	without losing the martingale properties. This is recorded in lemma \ref{lemma:stopmart}.
	\begin{lemma}
		\label{lemma:stopmart}[Stopped Martingale]
		Suppose that $(Z_i)_{i\in\mathbb{Ni}_0}$ is a martingale with respect to---and that 
		$\tau$ is a stopping time for--- the stochastic process $(X_i)_{i\in\mathbb{N}_0}$.
		Then the stochastic process $(T^\tau_i)_{i\in\mathbb{N}_0}$ defined by 
		$$
			Z^\tau_i = 
			\begin{dcases}
				Z_i & \tau \geq i \\
				Z_\tau & \tau < i
			\end{dcases}
		$$
		is a martingale with respect to $(X_i)_{i\in\mathbb{N}_0}$.
	\end{lemma}
	\begin{proof}
		Notice that $Z^\tau_i = Z^\tau_{i-1} + 1[\tau \geq i](Z_i - Z_{i-1})$.\\
		Then, we can see that $1[\tau \geq i]$ is a function of $(X_j)_{j=0}^{j=i-1}$,
		since $\tau$ is a stopping time for $(X_i)_{i\in\mathbb{N}_0}$. Thus, $Z_i^\tau$ is
		a function of $(X_j)_{j=0}^{j=i-1}$, satisfying the first axiom. The second and 
		third axioms are satisfied simply because $(Z_i)_{i\in\mathbb{N}_0}$ is a martingale.
	\end{proof}
	
\section{The Optional Stopping Theorem}




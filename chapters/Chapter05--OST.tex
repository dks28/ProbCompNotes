\chapter{The Optional Stopping Theorem}

In this chapter, we investigate the interaction of martingales with stopping times. As an example, 
we shall refer to the Gambler's Ruin process discussed previously. To remind the reader, the 
Gambler's Ruin is a stochastic process $(X_i)_{i\in\mathbb{N}_0}$ defined by $X_0 = k$, for some 
positive integer $k$, and $X_{i+1}$ is distributed uniformly over $\{-1, 1\}$ for any $i \in 
\mathbb{N}_0$. The usual interpretation (leading to the process's common name) is that $X_0$ is a 
starting capital for a gambler, and each subsequent $X_i$ indicates the profit made in the $i$th 
round of play. \\
Overall then, it makes sense to consider the capital the gambler possesses at each time step. 
Denote this by $(S_i)_{i\in \mathbb{N}_0}$, and note that we take $S_i = X_0 + \sum_{j=1}^{j=i}X_i$
for any $i$. Then, it should be reasonably obvious that $(S_i)_{i\in\mathbb{N}_0}$ is a martingale
with respect to $(X_i)_{i\in\mathbb{N}_0}$, since each $X_i$is bounded below and above by $k-i$ and
$k+i$, respectively (and the other axioms are obviously satisfied). 

We shall consider a few stopping strategies (which yield stopping times) for the gambler; for 
example the one we had considered previously, in which the gambler stops playing when either he 
goes bankrupt or when he has $N > k$ pounds. We have already explored his process to some extent, 
for example considering the expected capital the gambler would have by the time he stops playing. 
The main aim of this chapter is to establish whether the gambler could improve his prospects by 
changing his stopping strategy.

\section{Review of Gambler's Ruin Stopping Times}
	Recall that a stopping time for a a stochastic process $(X_i)_{i\in \mathbb{N}_0}$ is a 
	random variable $\tau$ such that any event $\tau = i$ can be defined depending only on $X_0
	\hdots X_i$. \\
	Now, suppose first that the gambler simply intends to play for a total of $n$ rounds, then 
	$\tau \coloneqq \lambda \omega . n$ is obviously a stopping time for 
	$(X_i)_{i\in\mathbb{N}_0}$. In fact, the expected capital of the gambler at time $\tau$ can
	easily be determined as $\mathbb{E}(X_0) = k$; this follows from a simple inductive 
	argument. 

	In fact, this co\"incides with the result we had previously obtained when considering the 
	Gambler's Ruin as a Markov chain, and follwoing a different stopping strategy: By the 
	result of lemma \ref{lemma:grprsucc}, we had that for our original stopping strategy (say 
	$\tau'$), $\mathbb{E}(S_{\tau'}) = 0 \cdot \frac{n-k}{n} + n \cdot \frac{k}{n} = k$.  

	\subsection{Stopped Martingales}
	We would like to be able to combine the ideas of matingales with those of stopping times.
	Luckily, we may simply use a construct which `stops' a martingale at a given stopping time,
	without losing the martingale properties. This is recorded in lemma \ref{lemma:stopmart}.
	\begin{lemma}[Stopped Martingale]
		\label{lemma:stopmart}
		Suppose that $(Z_i)_{i\in\mathbb{N}_0}$ is a martingale with respect to---and that 
		$\tau$ is a stopping time for--- the stochastic process $(X_i)_{i\in\mathbb{N}_0}$.
		Then the stochastic process $(Z^\tau_i)_{i\in\mathbb{N}_0}$ defined by 
		$$
			Z^\tau_i = 
			\begin{dcases}
				Z_i & \tau \geq i \\
				Z_\tau & \tau < i
			\end{dcases}
		$$
		is a martingale with respect to $(X_i)_{i\in\mathbb{N}_0}$.
	\end{lemma}
	\begin{proof}
		Notice that $Z^\tau_i = Z^\tau_{i-1} + 1[\tau \geq i](Z_i - Z_{i-1})$.\\
		Then, we can see that $1[\tau \geq i]$ is a function of $(X_j)_{j=0}^{j=i-1}$,
		since $\tau$ is a stopping time for $(X_i)_{i\in\mathbb{N}_0}$. Thus, $Z_i^\tau$ is
		a function of $(X_j)_{j=0}^{j=i-1}$, satisfying the first axiom. The second and 
		third axioms are satisfied simply because $(Z_i)_{i\in\mathbb{N}_0}$ is a martingale.
	\end{proof}
	
\section{The Optional Stopping Theorem}
	Let us now return to the question we originally set out to answer. Can the gambler devise 
	a strategy that will improve his expected fortune? This question is answered, under some 
	conditions, by the \emph{optional stopping theorem} (OST).
	\begin{theorem}[Optional Stopping Theorem]
		\label{theorem:ost}
		Let $(X_i)_{i\in\mathbb{N}_0}$ be a stochastic process with respect to which 
		$(Z_i)_{i\in\mathbb{N}_0}$ is a martingale. Let also $\tau$ be a stopping time for 
		$(X_i)_{i\in\mathbb{N}_0}$. Then, if
		\begin{enumerate}[(1)]
			\item the $Z_i$ are bounded, 
			i.e.\ $\exists c \in \mathbb{R}^+ . \forall i . |Z_i| < c$, or
			\item $\tau$ is bounded or
			\item $\mathbb{E}(\tau) < \infty$ and $\exists k \in \mathbb{R}^+ . 
			\mathbb{E}\left(|Z_{i+1}-Z_i| \big| (X_j)_{j=0}^{j=1}\right) < k$
		\end{enumerate}
		it holds that $\mathbb{E}(Z_\tau) = \mathbb{E}(Z_0)$.
	\end{theorem}
	To prove this theorem, we shall first state two pr{\ae}requisite theorems, ommitting 
	their proofs as they carries little relevance to our discussion. These theorems are 
	presented below.
	\begin{theorem}[Dominated Convergence Theorem]
		Suppose that a stochastic process $(X_n)_{n\in\mathbb{N_0}}$ converges pointwise to
		some random variable $X$ such that some random variable $Y$ has $\mathbb{E}(Y) <
		\infty$ and $|X_n| < Y$, then $\mathbb{E}(X) = \lim_{n\rightarrow \infty} 
		\mathbb{E}(X_n)$.
	\end{theorem}
	\begin{theorem}[Doob's Martingale Convergence Theorem]
		Suppose that $(M_n)_{n\in\mathbb{N}_0}$ is a martingale such that $\sup_{n\in
		\mathbb{N}}(\mathbb{E}(|M_n|^t)) < \infty$ for some $t>1$. Then there is a random 
		variable $M$ such that $\lim_{n\rightarrow \infty} M_n = M$. 
	\end{theorem}
	Using these, and the notion of a random variable being `integrable' if its expected value 
	is well-defined and finite, we may now prove the optional stopping theorem.
	\begin{proof}[Proof of theorem \ref{theorem:ost}]
		We conduct this proof by showing that under any of the conditions (1)-(3), the
		martingale $(Z_i)_{i\in \mathbb{N}_0}$ is dominated by some integrable random 
		variable $M$, which will allow us to derive the desired equality. \\
		Now, if condition (1) holds, then we can set $M$ to be the constant random 
		variable $M \coloneqq \lambda \omega . c$, which is clearly integrable, and
		by Doob's martingale convegence theorem the stopped martingale 
		$(Z^\tau_n)_{n\in\mathbb{N}_0}$ converges pointwise to the random variable 
		$Z_\tau$.\\
		If either condition (2) or condition (3) holds, then let $M \coloneqq |Z_0| 
		+ \sum_{i\in\mathbb{N}_0} |Z_{i+1}-Z_i| \cdot 1[\tau > i]$, which also clearly 
		dominates $(Z_n^\tau)_{n\in\mathbb{N}_0}$. Then, we have that if condition (2) 
		holds, then the series only has a finite number of nonzero terms, which implies
		that $M$ is integrable. If condition (3) holds, we have that
		\begin{align*}
			\mathbb{E}(M) &= \mathbb{E}(|Z_0|) + \sum_{i\in\mathbb{N}_0} \mathbb{E}(
			|Z_{i+1}-Z_{i}| \cdot 1[\tau>i]) \\
			&= \mathbb{E}(|Z_0|) + \sum_{i\in\mathbb{N}_0} \mathbb{E}\left(\mathbb{E}
			\left(|Z_{i+1}-Z_{i}|\big|(X_j)_{j=0}^{j=i}\right)\cdot1[\tau>i]\right) \\
			&\leq \mathbb{E}(|Z_0|) + k \sum_{i\in\mathbb{N}_0} \mathbb{P}(\tau > i) \\
			&= \mathbb{E}(|Z_0|) + k \mathbb{E}(\tau) \\
			&< \infty
		\end{align*}

		Thus any condition implies that the stopped martingale 
		$(Z_n^\tau)_{n\in\mathbb{N}_0}$ is dominated by a random varibale with finite and 
		well-defined expectation. Under conditions (2), (3) the random variable $Z_\tau$ is
		well-defined, too, so that under any of the conditions, $Z^\tau_n$ converges to 
		$Z_\tau$. \\
		Then, the dominated convergence theorem implies that $\mathbb{E}(Z_\tau) = 
		\lim_{n\rightarrow\infty}\mathbb{E}(Z_n^\tau)$. Since the stopped martingale is, 
		indeed, a martigale, we have that, $\forall n \in \mathbb{N}. \mathbb{E}(X_n^\tau) 
		= \mathbb{E}(X_0)$, so we have $\mathbb{E}(X_\tau) = \mathbb{E}(X_0)$.
	\end{proof}
	\begin{comment}
		It is worth pointing out that a similar result applies to super- and submartingales%
		, obviously with the relevant inequality instead.
	\end{comment}
	
	So with regards to the Gambler's Ruin example; the OST says that, if any one of the given 
	conditions holds, the gambler cannot improve on the stopping rule that stops his gambling 
	after a fixed period of time. So recall our set-up from before; the gambler starts with 
	$S_0 = k$ pounds, and stopswhen either he reaches $S_\tau = N$ pounds or he goes broke by 
	$S_\tau = 0$. Let us check that the OST applies here:
	\begin{itemize}
		\item Condition (1) does not hold, as for arbitrarily large $t$, $S_t$ is unbounded.
		\item Condition (2) is not satisfied either, as the gambler could conceivably remain
		between $0,N$ forever (even though the probability for that happening is, 
		of course, $0$.)
		\item If the OST is to apply, we will need to check that condition (3) does hold. 
		Luckily, the value $|S_{i+1} - S_i|$ is clearly bounded, so we need only check that 
		the expectation of $\tau$ is finite.
	\end{itemize}
	There is a trick we may apply here to avoid doing so. Consider the stopped martingale 
	$(S_i^\tau)_{\in\mathbb{N}_0}$, and recall that for $i\leq\tau$, $S_i = S_i^\tau$ 
	by definition, which suffices for us because the game is not interesting to the 
	gambler after he stops playing. Luckily, then, we can use the OST to conclude that
	$$
		\mathbb{E}(S_\tau) = \mathbb{E}(S_\tau^\tau) = \mathbb{E}(S_0^\tau) = 
		\mathbb{E}(S_0) = k
	$$
	since the stopeed martingale is bounded by $0 \leq S_i^\tau \leq N$.

	To conclude about this stopping rule, we remark that the OST implies that the gambler 
	simply cannot improve on his original stopping strategy whithout the circumstances 
	breaking the OST: This could occur if the gambler was psychic and could consult the future 
	to devise his strategy, which would break the `stopping time' condition on $\tau$. The 
	other way of breaking the OST would be to violate the conditions, which would imply (1) 
	the gambler having access to infinite credit (3) the gambler expecting to play indefinitely,
	since breaking condition (3) means that $\mathbb{E}(\tau) = \infty$. Breaking condition (2)
	is not exactly interesting as most elaborate stopping strategies may cause the gambler to 
	remain for an unbounded time.

	This is a sobering insight. Let us consider if we can at least establish how long the 
	Gambler may expect to remain in the casino. To this end, consider the stochastic process 
	$(Z_i)_{i\in\mathbb{N}_0}$ defined by $Z_i = S_i^2 - i$. Note that this is a martingale 
	with respect to $(X_i)_{i\in\mathbb{N}_0}$, since 
	\begin{itemize}
		\item clearly, $Z_i$ is a function of $(X_j)_{j=0}^{j=i}$,
		\item $Z_i$ is bounded by $-i\leq Z_i\leq(k+i)^2-i$, so $\mathbb{E}(Z_i)<\infty$ 
		and
		\item $\mathbb{E}(Z_{i+1} | (X_j)_{j=0}^{j=i})=\sfrac{1}{2}\cdot((S_i + 1)^2-i-1)+
		\sfrac{1}{2}\cdot((S_i-1)^2 - i-1) = S_i^2 - i = Z_i$.
	\end{itemize}
	Here, we can not apply the trick of the stopped martingale as before, since $\tau$ may be 
	arbitrarily large. Thus, we have to verify that the expectation of $\tau$ is indeed finite.
	Indeed, we have already proven this; modelling the Gambler's Ruin as the SRW over the $n$-%
	path, we have that $\mathbb{E}(\tau) \leq t_\mathrm{cov}(P_n) < 2n^2 <\infty$ by lemma 
	\ref{lemma:pathcovt}. Thus, the OST does apply by condition (3), and we can deduce
	$$
		k^2 = \mathbb{E}(Z_0) = \mathbb{E}(Z_\tau) = \mathbb{E}(S_\tau^2-\tau)
	$$
	so that $\mathbb{E}(\tau) = \mathbb{E}(S_\tau^2) - k^2$. We previously established that 
	$\mathbb{P}(S_\tau) = \sfrac{k}{N}$, so that 
	\begin{align*}
		\mathbb{E}(\tau) &= \mathbb{E}(S_\tau^2) - k^2 \\
		&= N^2 \cdot \sfrac{k}{N} \\
		&= k(N-k).
	\end{align*}

	Now, it may be of interest that, if the gambler were to play until he had made a net 
	profit of 1 pound, i.e.\ if he abided by the stopping time $\tau' = \inf(t\in\mathbb{N}_0 
	: S_t = k+1)$, then he would expect to remain in the casino forever. Consider why: If we 
	denote $\tau_{a,b} \coloneqq \inf(t\in\mathbb{N}_0: S-t = a \lor S_t = b)$, then clearly $
	\tau \geq \tau_{-a, k+1}$ for any $a \in\mathbb{N}$. Thus,
	$$
	\mathbb{E}(\tau) \geq \mathbb{E}(\tau_{-a, k+1} | X_0 = k) = 
	\mathbb{E}(\tau_{0,a+k+1}|S_0=k+a) = k+a
	$$ 
	Since this must hold for any $a$, $\mathbb{E}(\tau) = \infty$! The gambler would thus need a 
	very good credit card, and an inexhaustible amount of time to ensure a profit.


	From this result, of course, we may extend lemma \ref{lemma:pathcovt} to state
	\begin{lemma}
		\label{lemma:pathcovtext}
		For the SRW over the $n$-path $P_n$, the expected time it takes to reach one of the
		extremes, starting at $X_0 = k$, is $kn - k^2$. \par
		Furthermore, this implies that $t_\mathrm{cov}(P_n) = \max_k(n^2 +nk -k^2)$.
	\end{lemma}
	\begin{proof}
		We have already proven the first part of the lemma. The second part follows by the
		simple argument that the time it takes for the SRW to cover the graph is the time
		it takes to reach one extreme, followed by then hitting the other. This implies 
		that $t_\mathrm{cov}(P_n) = \max_k(nk-k^2) + h_{0,n} = \max_{k}(nk-k^2+n^2)$.
	\end{proof}

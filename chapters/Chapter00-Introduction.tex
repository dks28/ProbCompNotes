\chapter*{Introduction}
\addcontentsline{toc}{chapter}{Introduction}

	\paragraph{About this document.}
	This document is intended to serve as an exhaustive collection 
	of material for the Unit of Assessment ‘Probability and Computation’ in Part II of the
	Computer Science Tripos at the University of Cambridge, as lectured by Dr John Sylvester,
	Dr Nicolás Rivera, Dr Luca Zanetti and Dr Thomas Sauerwald.\par
	It  has  been  based  heavily off the lecture slides as provided by the lecturers for the academic year  
	2018/2019\footnote{Note that apparently the course content is subject to change},  but  
	aims  to  exhibit  more  thorough  explanations  and  provide
	some additional content where this may be helpful.  Some of the material also covered by
	the recommended reading for this Unit, Michael Mitzenmacher and Eli Upfal’s ‘Probability
	and Computing’ has been drawn from there rather than the lecture notes, but care has
	been taken to ensure all examinable material has been covered. \par
	Of course, some details may
	have been distorted by me misunderstanding them in the lecture slides.  Any inaccuracies in 
	mathematical arguments or other mistakes are thus, with high probability, mine; should such 
	mistakes be found I would be grateful should these be pointed out to me by e-mail at 
	dks28@cam.ac.uk. 
	
	\paragraph{About its content.}
	The course is an introduction to various kinds of randomised computation, and the important 
	mathematical concepts supporting it.  By ‘randomised compu-tation’, we understand any 
	algorithm or other instance of computation which has a randomised or probabilistic aspect.  
	Most commonly, this will feature making nondeterministic choices that will lead the algorithm 
	to return a correct result with some probability---this is of course only worth our while if 
	we gain a performance boost over a known deterministic algorithm where the result is always 
	correct.\par
	At other times, we consider determinsitic algorithms that operate on randomised inputs. One 
	example for such a computation is the well-known Bucket Sort algorithm, which, albe\"it 
	deterministic, assumes that the input of numbers to be sorted has been drawn from a known 
	probability distribution to achieve an expected improvement over the theoretical lower bound 
	for the run-time of a sorting algorithm.\par 
	This  course  will,  in  addition,  explore  many  further  mathematical  concepts  in  
	probability theory, such as random walks and Markov chains, and establish many important 
	mathematical techniques in working with probability theory, such as introducing a number of 
	probability bounds for deviations from the expected value of a distribution. \par
	Each of the aforementioned lecturers will focus on their own topics, so this document is 
	separated logically into four parts,  corresponding to each of the lecturers’ sections of the 
	course respectively.  As such, the four parts of these notes focus on Markov Chains, Martingales, 
	Random Graphs and Sublinear Algorithms, respectively.
